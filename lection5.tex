% \documentclass[9pt, compress, xcolor=table]{beamer}


% \usetheme{m}

% \usepackage{amsmath,amssymb,amsthm}
% \usepackage{latexsym}
% \usepackage{booktabs}
% \usepackage[scale=2]{ccicons}
% \usepackage{minted}
% \usepackage[english, russian]{babel}
% \usepackage{graphicx}
% \usepackage{xcolor}
% \usepackage{tabu} % https://ru.sharelatex.com/learn/Tables
% \DeclareGraphicsExtensions{.pdf,.jpg,.png}
% \graphicspath{{../images/}{./images/}}

% \colorlet{Mycolor1}{green!50!blue!50!}

% \usemintedstyle{trac}

% \title{Физические принципы оптической микроскопии сверхвысокого разрешения}
% \subtitle{осенний семестр, 2015}
% \author{ассистент, к.ф.-м.н. Шутова О.А.}
% \institute{МГУ им. М.В. Ломоносова, физический факультет}

% \begin{document}

% \maketitle

% \plain{}{Лекция 5. Нелинейная ближнепольная микроскопия}

% \begin{frame}{Нелинейная БП микроскопия}

% \begin{columns}[c]
% \column{6.5cm}

% Под нелинейной ближнепольной микроскопией понимают такой вид ближнепольной зондовой микроскопии, когда взаимодействие между электромагнитным полем с веществом образца и зонда носит нелинейный характер. 

% В первую очередь это микроскопия на основе генерации второй гармоники


% Нелинейная микроскопия равивается в группе профессора Маркуса Б. Рашке, Университет штата Колорадо.

% А также в работах профессора Анатолия В. Заяца, King's College London.

% \column{6cm}
% \begin{center}
% \includegraphics[width=0.9\textwidth]{gauss}
% \end{center}
% Возведение распределения поля в степень уменьшает величину его пространственной ширины.

% \end{columns}
% \end{frame}


% \begin{frame}{Микроскопия на второй гармонике}
% \begin{columns}[c]
% \column{6.5cm}
% \begin{center}
% \includegraphics[width=0.7\textwidth]{shg1}

% \includegraphics[width=0.65\textwidth]{shg2}
% \end{center}
 
% \column{6cm}
% \vfill Когда среда обладает центральной симметрией (а это 11 из 32 классов симметрии кристаллов) генерация ВГ из объема запрещена. Но это означает, что вторая гармоника исходит от поверхности.
 
% \hfill \break В случае сферической частицы (даже в отсутствии ЦС) излучение ВГ вперед будет подавляться деструктивной интерференцией.
 
% \hfill \break В ближнем поле генерация ВГ становится более сложным процессом в силу того, что мы уже не можем говорить о направлениях распространения света: преобладают процессы рассеяния. 
% \end{columns}
% \end{frame}

% \begin{frame}{Генерация ВГ от зонда}
% \begin{center}
% \includegraphics[width=0.6\textwidth]{shg3}
% \end{center}

% Для ЦС среды генерация ВГ обусловлена вкладом от поверхности, а также электрическим квадрупольным и магнитным дипольным
% \begin{multline*}
% P^{2\omega}(\vec r) = P^{2\omega}_{surface}(\vec r)+P^{2\omega}_{nonlocal}(\vec r)=\\=\chi^{(2)}_{surface}\otimes \vec E ^{(\omega)}(\vec r)\vec E^{(\omega)}(\vec r)\delta(z)+\chi^{(2)}_{bulk}\otimes \vec E ^{(\omega)}(\vec r)\nabla\vec E ^{(\omega)}(\vec r)
% \end{multline*}

% Важно учитывать корректировку за счет локального поля как для основной, так и для ВГ

% \scriptsize{См. И.Р.Шен <<Принципы нелинейной оптики>>, пар. 25.3. Использование нелинейных оптических эффектов для зондирования поверхности, с.464.}

% \end{frame}
% \begin{frame}{Генерация ВГ от зонда}
% \begin{columns}[c]
% \column{6.5cm}

% В общем случае у нас остаются ненулевыми следующие вомпоненты 
% \begin{equation*}
% \chi^{(2)}_{s,zii},\quad\chi^{(2)}_{s,izi}=\chi^{(2)}_{s,iiz},\quad\chi^{(2)}_{s,zzz}
% \end{equation*}

% Но в силу аксиальной симметричности зонда остаются три компоненты:
% \begin{equation*}
% \chi^{(2)}_{s,\perp\perp\perp},\quad\chi^{(2)}_{s,\parallel\perp\parallel},\quad\chi^{(2)}_{s,\perp\parallel\parallel}
% \end{equation*}

% S-пол. это локальный (дипольный) отклик за счет поверхностной $\chi^{(2)}_{surface}$, P- пол. - нелокальный отклик за счет $\chi^{(2)}_{bulk}$ в квадрупольном и магнито-дипольном приближнении
 
% \column{6cm}
% \begin{center}
% \includegraphics[width=0.9\textwidth]{shg5}
% \* $saggital_{in}-saggital_{out}$ наблюдение для разных поляризаций
% \end{center}
% \end{columns}

% \begin{center}
% \includegraphics[width=0.6\textwidth]{shg4}
% \end{center}

% \end{frame}

% \begin{frame}{Обобщение эксперимента}

% В геометрии \textcolor{blue}{$saggital_{in}-saggital_{out}$} (нет зеркальной симметрии):
% \begin{itemize}
% \item отклик $p_{in}p_{out}$ исходит от локального дипольного отклика за счет поверхностной квадратичной нелинейности
% \item отклик $p_{in}s_{out}$ и $s_{in}s_{out}$ происходят от нелокального квадрупольного и магнитного дипольного компонента квадратичной нелинейности
% \item отклик $s_{in}p_{out}$ отсутсвует
% \end{itemize}

% Таким образом, в этой случае играют роль все поляризации, мы можеи получать сигнал обладающий уникальными для зондовой геометрии свойствами.

% В геометрии  \textcolor{red!50!black}{$saggital_{in}-axial_{out}$} (в одной плоскости лежат $\vec k (\omega)$, $\vec k (2\omega)$ и ось зонда, подключается зеркальная симметрия относительно любой плоскости, рассекающей зонд вдоль его оси):
% \begin{itemize}
% \item отсутствует отклик на ВГ в направлении вдоль зонда
% \item отсутствует отклик для S-пол. на ВГ в обоих случаях 

% \end{itemize}
% \end{frame}
 
% \begin{frame}{Разделение сигналов ближнего и дальнего поля}

% Как и в линейной БП микроскопии разделение ближнепольного сигнала (несмотря на то, что он на частоте ВГ) и дальнепольного является нетривиальной задачей. И так же используется свойство ближнего поля зависеть от расстояния на масштабах много меньше длины волны (дальнее поле на таких масштабах от расстояния не зависит). 

% Дрожание зонда равно кривизне его кончика, т.е. приблизительно 15 нм.

% \begin{center}
% \includegraphics[width=\textwidth]{shg18}
% \end{center}

% Сигнал отклика равен разности сигналов в положени \emph{ tip in} и \it{tip out}.

% \end{frame}

% \begin{frame}{Островки алюминия на стеклянной подложке}

% \begin{center}
% \includegraphics[width=0.7\textwidth]{shg7}
% \newline {\scriptsize Падающее поле p поляризовано, ВГ наблюдается без разрешения поляризации}
% \end{center}

% В случае, когда квадратичная восприимчивость образца много меньше, чем зонда, можно считать, что процесс генерации ВГ происходит только в зонде (как в случае стеклянной подложки). 
% В общем случае:

% \begin{equation*}
% I(2\omega)\propto |\sum_i\chi^{(2)}_{i}\alpha^{eff}_i(\omega)^2+\chi^{(2)}_{t}\alpha^{eff}_t(\omega)^2|^2 E^4(\omega)
% \end{equation*}
% причем эффективные поляризуемости зависят, каждая от поляризумости и зонда, и образца.
% \end{frame}

% \begin{frame}{Аналогичный эксперимент для частичек золота}
% \begin{columns}
% \column{7cm}
% \begin{center}
% \includegraphics[width=\textwidth]{shg10}
% \end{center}
% \column{5cm}
% Усиление сигнала ВГ над частичками золота достигает двух раз. Некоторые объекты, видимые на топографии, отсутствуют в сигнале ВГ. Следует предположить, что это пылинки или др. дефекты.

% Усиление для золота примерно в два раза лучше, чем для алюминия.
% \end{columns}
% \end{frame}
% \plain{}{Исследование сегнетоэлектрических доменов с помощью ближнепольной микроскопии на ВГ}
% \begin{frame}{Сегнетоэлектрики}
% \begin{columns}
% \column{6.5cm}
% \begin{center}
% \includegraphics[width=0.7\textwidth]{shg27}
% \end{center}
% \column{6.5cm}
% \begin{center}
% \includegraphics[width=0.8\textwidth]{shg28}
% \end{center}

% \end{columns}

% Самый известный сегнетоэлектрик - титанат бария. При температуре Кюри атом титатна скачком смещается из центра на небольшую величину, симметрия переходит из кубической в тетрагональную. Центральному катиону становится энергетически более выгодно сблизиться с одним из анионов, чем быть на равных расстояниях со всеми.  

% Исследование сегнетоэлектрических доменов - важное направление в технологии создания оперативной твердотельной памяти (Ferroelectric RAM, FeRAM).

% \end{frame}

% \begin{frame}{Доменная стенка}
% \begin{center}
% \includegraphics[width=0.8\textwidth]{shg12}
% \end{center}

% \begin{equation*}
% \eta=\pm \sqrt{-\frac{A}{B}}\text{th}\frac{xz}{r_c},\quad \text{где}\quad r_c = \left(\frac{\delta}{A}\right)^{1/2}
% \end{equation*}


% Эта величина может быть различной для одного и того же материала в разных условиях, варьируется от единиц до десятков ангстрем. В отличие от ферромагнитных доменных стенок, которые состоят из сотен кристаллических ячеек.

% \end{frame}

% \begin{frame}{Некоторые понятия}

% \begin{itemize}
% \item \textcolor{red!50!black}{Мультиферроики} -  (или сегнетомагнетиками в советской литературе) называют материалы, в которых сосуществуют одновременно два и более типов «ферро» упорядочения.

% \item \textcolor{red!50!black}{Перовскиты} - вещества в основе строения которых лежит кубическая решетка, как у титаната бария с актионно-анионной групой.

% \item \textcolor{red!50!black}{Манганиты} - вещества на основе марганца $A_x Mn O_3$, могут иметь решетку типа первоскитов, а могут иметь гексагональную решетку, A -редкоземельный металл, лантан, скандий, иттрий и лантаноиды (57-71).

% \item \textcolor{red!50!black}{КМС (коллосальное магнетосопротивление)} -  квантовомеханический эффект заключающийся в сильной зависимости электрического сопротивления материала от величины внешнего магнитного поля. Наблюдается в манганитах.

% \item \textcolor{red!50!black}{$Y Mn O_3$} - недавно было обнаружено его свойство быть мультиферроиком, герой нижеследующего изложения.

% \end{itemize}

% Мультиферроики играют важную роль в разработке энергонезависимой (non volatile) памяти.

% \end{frame}

% \begin{frame}{Структура}
% $Y Mn O_3$ имеет гексагональную структуру
% \begin{columns}
% \column{10cm}
% \begin{center}
% \includegraphics[width=0.8\textwidth]{shg11}
% \end{center}
% \column{2.5cm}
% \centering
% \textcolor{blue!50!black}{Неисчезающие $\chi^{(2)}$ в классе $C_{6v}$}:


% $\chi^{(2)}_{xzx}=\chi^{(2)}_{xzx}$

% $\chi^{(2)}_{xxz}=\chi^{(2)}_{yyz}$

% $\chi^{(2)}_{xxz}=\chi^{(2)}_{zyy}$

% $\chi^{(2)}_{zzz}$

% \end{columns}
% Из класса симметрии $6/mmm$ (или $D_{6h}$, гексагональная группа, 24-й порядок) переходит $6mm$ (или $C_{6v}$, гексагональная группа, 12-й порядок, теряет центральную симметрию), $T_c=913^{\circ}$

% \end{frame}
% \begin{frame}{ВГ от $Y Mn O_3$}

% \begin{columns}
% \column{6.5cm}
% \begin{center}
% \includegraphics[width=0.8\textwidth]{shg13}

% \includegraphics[width=0.7\textwidth]{shg15}
% \end{center}
% \column{6.5cm}
% \begin{center}
% \includegraphics[width=0.9\textwidth]{shg14}
% \begin{equation*}
% P_x^{(2)}(2\omega) = 2\epsilon_0\chi_{xxz}E^x(\omega)E^z(\omega)
% \end{equation*}
% \begin{equation*}
% P_y^{(2)}(2\omega) = 2\epsilon_0\chi_{xxz}E^y(\omega)E^z(\omega)
% \end{equation*}
% \begin{equation*}
% P_x^{(2)}(2\omega) = 2\epsilon_0\chi_{xxz}E^x(\omega)E^z(\omega)
% \end{equation*}

% Для величины сигнала:
% \begin{equation*}
% I_{||}^{SH} \propto \chi_{zzz}^{(2)}E^4 \cos^6 \alpha +\chi_{zyy}^{(2)}E^4 \sin^4 \alpha
% \end{equation*}
% \begin{equation*}
% I_{cross}^{SH} \propto \chi_{zyy}^{(2)}E^4 \cos^6 \alpha
% \end{equation*}
% \end{center}

% \end{columns}

% \end{frame}

% \begin{frame}{Усиление сигнала ВГ}

% Общее изучение характера усиления ближнепольного сигнала второй гармоники 

% \begin{center}
% \includegraphics[width=0.8\textwidth]{shg16}

% \end{center}

% Для p-поляризованного падающего поля под углом $70 ^{(o)}$ (в таком случае $E_x = E \sin 70^{o} = 0.94 E$, а $E_y = E \cos 70 ^{o} = 0.34 E$).

% \end{frame}

% \begin{frame}{Разрешение отклика по поляризации}

% Как отделить сигнал второй гармоники, исходящий от сегнетоэлектрических доменов, от второй гармоники, генерируемой зондом?

% Необходимо, апеллируя к приведенным выше исследованиям зонда, выбрать область, где сигнал от зонда минимален ($p_{in} - s_{out}$). Здесь как раз будет играть роль $\chi_{zxx}^{(2)}$ (поле поляризовано вдоль оси $z$ образца).

% \begin{center}
% \includegraphics[width=0.7\textwidth]{shg17}

% \end{center}

% \end{frame}

% \begin{frame}{Визуализация сегнетоэлектрических доменов}

% \begin{center}
% \includegraphics[width=0.8\textwidth]{shg19}

% \end{center}

% Слева мы видим картину, интерференционного сложения дальнепольного и ближнепольного сигнала второй гармоники, условия для которой меняются при переходе через доменную стенку. Справа мы находимся в области доминирования сигнала ВГ от зонда.


% \end{frame}

% \begin{frame}{Визуализация сегнетоэлектрических доменов}

% \begin{center}
% \includegraphics[width=0.8\textwidth]{shg20}

% \end{center}

% \end{frame}
% \begin{frame}{Рассеяние без БП контраста}

% \begin{center}
% \includegraphics[width=0.8\textwidth]{shg22}

% \end{center}

% Чувствительность ближнего поля к форме, размеру, шероховатости его поверхнсти, приводит к тому, что далеко не каждый экземпляр зонда, дает хороший ближнепольный сигнал.


% \end{frame}

% \begin{frame}{Разрешающая способность}

% \begin{columns}
% \column{6.5cm}
% \begin{center}
% \includegraphics[width=0.8\textwidth]{shg23}

% \includegraphics[width=0.6\textwidth]{shg25}

% \includegraphics[width=0.7\textwidth]{shg24}
% \end{center}

% \column{6cm}

% В фокусе (2.5 $\mu$м) много доменов участвуют в общей поляризации, связанной с СЭ
% \begin{equation*}
% \vec P^{(2)}_{FF}(2\omega) = \sum_{i=1}^{N_1}(\vec P^{(2)}_{fel})_i(2\omega)+ \sum_{i=1}^{N_2}(\vec P^{(2)}_{fel})_i(2\omega)
% \end{equation*}
% БП ВГ локализуется исключительно в области под зондом
% \begin{equation*}
% \vec P^{(2)}_{loc}(2\omega) \propto L(2\omega)L(\omega)L(\omega)a P^{(2)}_{fel})_i(2\omega)
% \end{equation*}
% Интерференционная картина:
% \begin{equation*}
% I(2\omega) \propto \left|P^{(2)}_{FF})_i(2\omega) \pm \vec P^{(2)}_{loc}(2\omega)\right|^2
% \end{equation*}

% {\scriptsize 10 антипараллельных доменов из 20 диполей каждый, расстояние между диполями 5 нм, в зависимости от расстояния от зонда 6 нм (синий), 10 нм (красный), 18 нм (черный).}

% \end{columns}
% \end{frame}
% \end{document}
